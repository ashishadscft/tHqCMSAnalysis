
\thispagestyle{empty}
 \vspace*{2.2cm}
\begin{center}
\Large \textbf{Resumen}
\end{center} 
\vspace{1.5cm}

%\renewcommand{\absnamepos}{empty} % originally center
\begin{center}
	\justifying
Presento un estudio sobre la producci\'on de un boson de Higgs y un top quark ($tH$) en un canal de 2 muones con el mismo signo $\mu^\pm \mu^\pm$ usando datos publicados por el experimento CMS en el CERN. Estudiando este proceso, se explora este mecanismo de producci\'on del boson de Higgs que a\'un no se ha detectado experimentalmente. La sensibilidad esperada es calculada usando un grupo de datos tipo Asimov para 35.9 fb$^{-1}$ para colisiones de prot\'on-prot\'on y esos datos son extrapolados a niveles altos de luminosidad, de acuerdo a la fase de alta luminosidad en el LHC. La sensibilidad de la se\~nal es tambi\'en estudiada para el modelo con un acoplamiento invertido de Yukawa $k_t$=-1, donde $k_t$ es el par\'ametro de acoplamiento de top-Higgs, para compararlo con el Modelo Est\'andar.
\end{center}
