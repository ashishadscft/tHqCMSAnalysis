
\chapter{Conclusion and outlook}
In this work, we explored the sensitivity to the tH production process.
The $tH$ process is one of the processes that comes from the proton-proton collisions in the LHC. This process has not been observed experimentally. Recently CMS published a search for $tH$ process using 35.9 fb$^{-1}$ from Run 2 in 2018. Using this information, it was studied the expected sensibility using an Asimov data and systematic uncertainties for final states of same sign pairs of muons \\

In the SM case, the signal strength for the signal $tH$ is 2.2 that is very low in comparison to
the backgrounds, resulting in large uncertainties. The uncertainty is around eight times than the signal strength. 
For the inverted coupling scenario ($k_t$=-1), the uncertainty for the model after the fit was better than the
SM case, this is due to the larger expected number of signal events. In this case 26.2 with an uncertainty of around 50$\%$.
\\

We also studied the sensitivity for larger integrated luminosities expected in future runs of the LHC. We find that for the SM scenario the expected uncertainty even at the largest luminosities is not enough to observe the signal and only an upper limit can be placed. In the case of the inverted coupling scenario, a possible signal can be observed with 10$\%$ uncertainty at 3000 fb$^{-1}$.\\

In the SM, same sign $\mu\mu$ final states of the $tH$ process, it shows that is impossible to give evidence of the existence of a Higgs boson due to high uncertainties. The expected uncertainty even at the largest luminosities is not enough to observe the signal and only an upper limit can be placed. 
For the $k_t$=-1, the uncertainty is low due to higher number of events for tH process and it is possible to give evidence of a Higgs
boson, but this model is purely theoretical.\\

However, these results use only the dimuon channel, the sensitivity can be improved by combining more search channels. The search can include channels, such as three lepton in the final state or a pair of leptons with different sign. By adding more channels, the number of events increases and can give us better statistics. 

