\chapter{Conclusion and outlook}

\chapter{Conclusion and outlook}
In this work, we explored the $tH$ processes and its different decay rates, including the Higgs mechanism that generate the bosons involved in theses processes. 
The $tH$ process is one of the processes that comes from the proton-proton collisions inside the accelerators. But due to the maintenance schedule of the LHC, many researchers uses Monte Carlo simulation and real data or even only MC for analyze the results and create predictions. In our case, by using MC simulations with an Asimov data, it was obtained a fit for the Asimov data with a model of signal and background events with systematic uncertainties according to the decay of processes to a pair of same sign muons $\mu$. \\

In the results for SM case, we got for the actual luminosities that the signal strength for the signal $tH$ is very low and that generates a signal-data ratio very small in comparison to the backgrounds, generating great uncertainties. For that, it was used extrapolations of the model by increasing the luminosity according to the LHC schedule, and discovered that by increasing the number of events, the signal uncertainties was decreasing, and improving the fit. In the likelihood scans in higher luminosities, the curves were closing, indicating that the model and the data incompatibility was decreasing in lower values of $\mu$. 
\\

However, for the $k_t$=-1, the uncertainty for the model after the fit was less than the SM case, so by increasing the luminosity the compatibility of model and the data was decreasing for higher luminosities, given that the $\mu$ in the likelihood scan w


All the research made in the CMS at CERN has let us discover the nature of the matter and try to answer questions about the origin of the universe, including the origin of the mass of the things. \\


