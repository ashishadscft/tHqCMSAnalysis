

\chapter{Conclusion and outlook}
In this work, we explored the $tH$ processes and its different decay rates. 
The $tH$ process is one of the processes that comes from the proton-proton collisions in the LHC. This process has not been observed experimentally. Recently CMS published a search for $tH$ process using 35.9 fb$^{-1}$ from Run 2 in 2018. \\

MC simulation and real data or even only MC can be used to study sensitivity different signal processes. In this work we used the information from CMS to study the expected sensitivity using an Asimov data and systematic uncertainties in the final state of a pair of same sign muons. \\
In the SM case, the signal strength for the signal $tH$ is very low in comparison to
the backgrounds, resulting in large uncertainties.
For the inverted coupling scenario ($k_t$=-1), the uncertainty for the model after the fit was better than the
SM case, this is due to the larger expected number of signal events. However, these results use only the dimuon channel, the sensitivity can be improved by combining more search channels.
\\

We also studied the sensitivity for larger integrated luminosities expected in future runs of the LHC. We find that for the SM scenario the expected uncertainty even at the largest luminosities is not enough to observe the signal and only an upper limit can be placed. In the case of the inverted coupling scenario, a possible signal can be observed with 10$\%$ uncertainty at 3000 fb${-1}$.\\




