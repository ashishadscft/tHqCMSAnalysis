%% 
%% Copyright 2016 Icm Ltd
\documentclass[final,3p]{CSP}
\usepackage{amssymb}
\usepackage{changepage}
\begin{document}

\begin{frontmatter}

\title{THESIS PROJECT PROPOSAL}

\author[]{Ashish Sehrawat}
%\author[]{Director: Dr. Jose Feliciano Benitez Rubio}

%\author[mysecondaryaddress]{Global Customer Service\corref{mycorrespondingauthor}}
%\cortext[mycorrespondingauthor]{Corresponding author}
%\ead{support@Icm.com}

\address[mymainaddress]{Universidad de Sonora}
%\address[mysecondaryaddress]{360 Park Avenue South, New York}

%\begin{keyword}\rm
%\begin{adjustwidth}{2cm}{2cm}{\itshape\textbf{Keyword:}}  
%The paper...
%\end{adjustwidth}
%\end{keyword}


\begin{keyword}\rm
\begin{adjustwidth}{2cm}{2cm}{\itshape\textbf{Title:}}  
Study of the Higgs boson production in association with a single top quark in proton collisions at the Large Hadron Collider.
\end{adjustwidth}
\end{keyword}

\begin{abstract}\rm
% \begin{adjustwidth}{2cm}{2cm}{
\itshape\textbf{Abstract:}
The abstract
%\end{adjustwidth}
\end{abstract}


\end{frontmatter}

\section{BACKGROUND}
\cite{Dirac1953888}

-SM, Higgs, couplings, previous Higgs measurements, previous tH searches

The standard model (SM) of particle physics is so far the best theoretical model to describe the interaction of elementary 
particles using three of the four fundamental forces of nature which are electromagnetic force, strong nuclear force and the weak
nuclear force. Gravitational force is neglected as the strength of this force is very weak at the scales over which elementary 
particle interact with each other. The standard model (SM) of particle physics is divided into two categories, bosonic sector 
and fermionic sector. Bosonic sector contain particles called bosons which mediate the fundamental forces of nature and the
fermionic sector contain particles called fermions which make up all the matter in our universe. SM has three generations of 
matter (fermions) particles.The first generation of fermions consists of up (u) quark, down (d) quark, electron and electron 
neutrino,second generation consist of charm (c) quark, strange (s) quark, muon and muon neutrino and the third generation of matter 
particles has top(t) quark,bottom(b) quark, tau and tau neutrino. The bosonic sector consist of gauge bosons like gluon, photon, 
W^+ W^- Z^0 which mediate strong force, electromagnetic force and weak force respectively. There is one more particle in the 
standard model called the Higgs Boson which gives mass to SM particles via electroweak symmetry breaking mechanism. Higgs boson 
can be produced at the particle colliders like the Large Hadron Collider (LHC) in Geneva, Switzerland. 

There are various production modes of Higgs boson like the gluon gluon fusion (ggF), vector boson fusion (VBF), Higgs production 
in association with vector boson (VH, V=W or Z), Higgs production with top quark and top anti-quark (ttH) and Higgs production with 
a single top quark and a quark jet (tHq). Each production channel has its own importance to probe the properties and the coupling 
strength of Higgs boson to fermions and bosons. The main production mode of Higgs bosons at LHC is gluon gluon fusion (ggF). As 
bosons like gluons and photons are massless, they do not interact directly with the Higgs boson and processes like ggF or Higgs 
boson decay to pair of photons are not possible at tree level and can only proceed via loop diagrams which involve W boson or top 
quark in the loop. The Higgs boson can also decay to other particles and these decay channels have different branching fractions. 
Higgs boson can decay to a pair of W bosons, Z bosons, photons, bottom quarks, muons and electrons. Higgs boson can also decay to 
Z boson and a photon. It is also interesting to investigate some invisible decay modes of Higgs boson which can be used to put 
upper bounds on dark matter-nucleon scattering cross section like in the Higgs-portal dark matter models. 

The Standard Model (SM) of particle physics has successfully described most of the experimental data till now but a very large number of
free-parameters like fine structure constant \alpha, Weinberg angle or weak mixing angle \theta_W, the coupling constant of strong 
interaction, electroweak symmetry breaking energy scale, Higgs potential self coupling or the Higgs mass, three mixing angles and the 
CP-violating phase of the CKM matrix which tells us how quarks of different color charge mix with each other, nine yukawa couplings 
which determine the mass of nine charged fermions and fine tunings related to the origin of masses robustly suggests new physics beyond 
the SM (BSM).There are lot of specific beyond BSM theories and most of these models involve heavy fields. In order to identify which new 
physics lies beyond the electroweak (EW) scale, the new parameters of such theories may be constrained by the actual, low energy, 
experiments. This approach requires studying each model individually, and calculating every possible observable. There is another 
approach immitating Fermi’s treatment of beta decay which consists of considering the SM as the first order approximation of the actual 
theory and by completing it with a series of higher dimensional operators. When the electroweak symmetry breaking takes place well below 
the mass of the new particles, the BSM physics is taken into account at the EW scale and below by adding higher dimensional operators to 
the SM Lagrangian. They are built out of SM fields and supposed to be invariant under its gauge group. Those operators are the low-
energy remnant of the high energy theory. This approach does not pretend to guess the complete high-energy model and it is based solely 
on the symmetry of the  theory. The operators are general and the only model dependence is encoded in the size of the operator 
coefficients, which is to be set from experiments. 

The Higgs boson and fermions (quarks and leptons) coupling will deviate from the standard model predictions if these higher dimensional 
operators are present in low energy effective SM theory. We can consider anomalous Higgs and Higgs-gauge effective dimension 6 operators 
like \frac{1}{3}(\phi^{\dagger}\phi)^3, \frac{1}{2} \partial_{\mu} (\phi^{\dagger} \phi) \partial^{\mu}(\phi^{\dagger} \phi) and 
(\phi^{\dagger} \phi)(D_{\mu} \phi)^{\dagger} (D^{\mu} \phi) to calculate the deviation of Higgs boson-fermion couplings from SM. The 
first operator shift the minimum of the Higgs potential. Also as there is a rescaling of the Higgs field due to the introduction of 
these operators in the potential term, Higgs-fermion (quarks and leptons) couplings are modified. In the effective beyond standard model 
theory, there are dimension 5 and dimension 6 operators. Dimension 5 operators are odd under baryon minus lepton number symmetry and 
dimension 6 operator dominate baryon minus lepton number symmetry conservation processes. 

The production of Higgs boson in association with single top quark is one of the rare Higgs boson production mode. As the top 
quark is the heaviest fundamental particle in the standard model, due to its large mass, the top quark decay before 
hardronization which allow the possibility to reconstruct top quark from its decay products unlike the lighter quarks which 
undergo hardronization and are seen as bundles of particles in detector called jets. The most probable decay of top quark is into 
bottom (b) quark and W boson. Using b-jet tagging algorithms, the jet originating from b-quark can be reconstructed to identity 
the top quark. In SM, the coupling of the Higgs boson to fermions like quarks and leptons is proportional to the mass of the 
fermions. Thus heavy quarks like top, bottom and charm couple strongly to Higgs boson which means that out of all known quarks, 
top quark couple most strongly to the Higgs boson. So, it has a large value of Yukawa coupling y_t. That is why the Yukawa 
coupling of the top quark with the Higgs boson y_t has lot of importance as any deviation from standard model prediction might 
give an indication of new physics. The two processes at LHC which will allow us to directly probe the Yukawa coupling between 
Higgs Boson and top quark are Higgs boson production in association with top quark pair (t\bar{t} H) via strong interaction and 
the production of single top quark with a Higgs boson (tH). As tH occur via electroweak interaction, it is more rare than 
t\bar{t}H production. The Higgs boson in the tH channel can be radiated off by a W boson or by a top quark. These two processes 
in standard model have destructive interference which allow us to probe the relative sign between the coupling of top quark with 
Higgs boson y_t and the coupling of W boson to the Higgs boson. If the Yukawa coupling between top quark and Higgs boson y_t 
deviate from the SM prediction or even if y_t has a negative sign, both would cause a strong increase in the tH production cross 
section which is a special property of tH production channel making it an interesting production channel to probe using LHC 
2016, 2017 and 2018 data.  


\section{PROPOSAL}
- what we will do:  search for tH events, using which channels.

It is proposed to study the production channel of the Higgs boson associated with a single top quark in the Large Hadron Collider of the 
CERN laboratory. The measurement of this process complements other measurements of the yukawa coupling parameter of the top quark and 
Higgs boson, is part of the physics program of CMS international collaborations and ATLAS and it has not been observed so far. It is 
proposed to study sensitivity with the current data in the channel where the events are identified with two muons of the same sign. The 
impact of uncertainties will be studied and projections will be made to the future phases of the LHC.

\section{GENERAL OBJECTIVE}
- motivation, tH production crossection, sign of k_t, search for BSM, ..


%\section{SPECIFIC OBJECTIVES}
%- details 

\section{HYPOTHESIS}
- what we expect to find

\section{METHODOLOGY}
- methods used in the search: which dataset,  lepton reconstruction, selections, jets, BDT, backgrounds, statistical analysis

\section{EXPECTED RESULTS}
- a limit, also predictions for future runs

\bibliographystyle{unsrt}
\cleardoublepage
\bibliography{paper}


\appendix
\section{SKILLS THAT WILL BE DEVELOPED}

\section{CALENDAR OF ACTIVITIES}



\end{document}

