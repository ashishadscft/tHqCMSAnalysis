%% 
%% Copyright 2016 Icm Ltd
\documentclass[final,3p]{CSP}
\usepackage{amssymb}
\usepackage{changepage}
\begin{document}

\begin{frontmatter}

\title{THESIS PROJECT PROPOSAL}

\author[]{Ashish Sehrawat}
%\author[]{Director: Dr. Jose Feliciano Benitez Rubio}

%\author[mysecondaryaddress]{Global Customer Service\corref{mycorrespondingauthor}}
%\cortext[mycorrespondingauthor]{Corresponding author}
%\ead{support@Icm.com}

\address[mymainaddress]{Universidad de Sonora}
%\address[mysecondaryaddress]{360 Park Avenue South, New York}

%\begin{keyword}\rm
%\begin{adjustwidth}{2cm}{2cm}{\itshape\textbf{Keyword:}}  
%The paper...
%\end{adjustwidth}
%\end{keyword}


\begin{keyword}\rm
\begin{adjustwidth}{2cm}{2cm}{\itshape\textbf{Title:}}  
Study of the Higgs boson production in association with a single top quark in proton collisions at the Large Hadron Collider.
\end{adjustwidth}
\end{keyword}

\begin{abstract}\rm
% \begin{adjustwidth}{2cm}{2cm}{
\itshape\textbf{Abstract:}
The abstract
%\end{adjustwidth}
\end{abstract}


\end{frontmatter}

\section{BACKGROUND}
\cite{Dirac1953888}

-SM, Higgs, couplings, previous Higgs measurements, previous tH searches

The standard model (SM) of particle physics is so far the best theoretical model describing the interaction of elementary particles using three of the four fundamental forces of nature which are electromagnetic force, strong nuclear force and the weak nuclear force. Gravitational force is neglected as the strength of this force is very weak at the scales over which elementary particle interact with each other. The standard model (SM) of particle physics is divided into two categories, bosonic sector and fermionic sector. Bosonic sector contain particles called bosons which mediate the fundamental forces of nature and the fermionic sector contain particles called fermions which make up all the matter in our universe. SM has three generations of matter(fermions) particles.The first generation of fermions consists of up quark, down quark, electron and electron neutrino, second generation consist of charm quark, strange quark, muon and muon neutrino and the third generation of matter particles has top quark, bottom quark, tau and tau neutrino. The bosonic sector consist of gauge bosons like gluon, photon, W^+,W^-,Z^0 which mediate strong force, electromagnetic force and weak force respectively. There is one more particle in the standard model called the Higgs Boson which gives mass to SM particles via electroweak symmetry breaking. Higgs boson can be produced at the particle colliders like the Large Hadron Collider in Geneva. There are various production modes of Higgs boson like the vector boson fusion(VBF), higgs production in association with vector boson(VH, V=W or Z), Higgs production with top quark and top anti-quark(ttH) and Higgs production with a single top quark and a quark jet(tHq). Each production channel has its own importance to probe the properties and the coupling strength of Higgs boson to fermions and bosons. The Higgs boson can also decay to other particles and these decay channels have different branching fractions. Higgs boson can decay to a pair of W bosons, Z bosons, photons, bottom quarks, muons and electrons. It can also decay to Z boson and a photon. It is also interesting to investigate some invisible decay modes of Higgs boson which can be used to put upper bounds on dark matter-nucleon scattering cross section like in Higgs-portal dark matter models.

\section{PROPOSAL}
- what we will do:  search for tH events, using which channels.

\section{GENERAL OBJECTIVE}
- motivation, tH production crossection, sign of k_t, search for BSM, ..

%\section{SPECIFIC OBJECTIVES}
%- details 

\section{HYPOTHESIS}
- what we expect to find

\section{METHODOLOGY}
- methods used in the search: which dataset,  lepton reconstruction, selections, jets, BDT, backgrounds, statistical analysis

\section{EXPECTED RESULTS}
- a limit, also predictions for future runs

\bibliographystyle{unsrt}
\cleardoublepage
\bibliography{paper}


\appendix
\section{SKILLS THAT WILL BE DEVELOPED}

\section{CALENDAR OF ACTIVITIES}



\end{document}

